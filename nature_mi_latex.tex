
% Results Section for Nature Machine Intelligence

\section{Results}

\subsection{Universal Scaling Laws}

Analysis of 11 reinforcement learning configurations generating 
5.4 million molecules revealed a clear scaling relationship 
between the exploration parameter $\sigma$ and chemical space navigation efficiency. 
The champion configuration (Run13c, $\sigma = 60$) achieved 
316 gold standard molecules, representing a 
10.313\% success rate.

The exploration efficiency $E$ follows a biphasic relationship with $\sigma$:

\begin{equation}
E(\sigma) = A_1 \exp\left(-\frac{(\sigma - \sigma_1)^2}{2w_1^2}\right) + 
            A_2 \exp\left(-\frac{(\sigma - \sigma_2)^2}{2w_2^2}\right)
\label{eq:biphasic}
\end{equation}

where $\sigma_1 \approx 60$ represents the exploitative regime and 
$\sigma_2 \approx 120$ represents the balanced exploration regime.

\begin{table}[h]
\centering
\caption{Summary of key configurations}
\begin{tabular}{lcccc}
\hline
Configuration & $\sigma$ & Molecules & Gold Standard & Efficiency \\
\hline
run13c & 60 & 306,415 & 316 & 543.2 \\
run14a & 150 & 1,318,183 & 113 & 806.4 \\
run15 & 150 & 1,069,649 & 112 & 816.9 \\
run9_t1200 & 120 & 1,096,589 & 96 & 819.2 \\
run13b & 120 & 806,382 & 89 & 828.6 \\
\hline
\end{tabular}
\label{tab:results}
\end{table}

The failed configurations (Run3, Run4, Run5) with zero gold standard molecules 
despite generating > 200,000 molecules collectively, demonstrate the importance 
of appropriate $\sigma$ selection. These ``failures'' map the boundaries of 
accessible chemical space, providing valuable negative information.
